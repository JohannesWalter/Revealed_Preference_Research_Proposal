Subsection 1) Data collection
describe how he did it and what kind of data you need to collect (i.e. recognize when asked for cookies and what the default cookie setting is)

Also describe a little bit how cookies actually work

What kind of data is required?




\begin{itemize}
    \item Data needs to have an opt-in and an opt-out possible framing. This should be given in the cookie context: 
    \begin{itemize}
        \item What does the GDPR say exactly?
        \item Either it says all cookies always must be disabled by default \dots 
        \item Or it varies only by how the website presents the choice. What are the options? 
    \end{itemize}
    \item I need to observe whether individual i chooses only the technically necessary cookies or all the cookies
    \item I also need to observe how the choice is presented: 
\end{itemize}

A) Data
Komplettes Deaktivieren der Cookies in Firefox oder Cookies deaktivieren für nutzungsbasierte Online-Werbung
https://www.bild.de/wa/ll/bild-de/privater-modus-unangemeldet-54578900.bild.html

By using our site, you acknowledge that you have read and understand our Cookie Policy, Privacy Policy, and our Terms of Service
https://tex.stackexchange.com/questions/823/remove-ugly-borders-around-clickable-cross-references-and-hyperlinks

I observe whether individual i accepts all cookies or only the technically necessary ones $(Y_{i})$, and whether the default is opt-in $ (D_{i} = 0) $ or opt-out $ (D_{i} = 1) $ at the date of visitng the website. 
Through an additional survey, we also observe age, sex and race for each employee.

B) Recovery of Consistent Preferences

\begin{itemize}
    \item Under assumptions A1 - A4, proposition 1 allows us to identify the preferences of the consistent visitors
    \item A1) requires that the preferences over the cookie choice do not depend on opt-in or opt-out.
    \item A2) Frame exogeneity
\end{itemize}

Under A)1 to A4), we identify the preferences of the consistent decision makers.
What you get is the preference of the consistent decision-makers.

C) Recovery of the Population Preferences

Subsection 2) Identification and identification of popuilation preferences

Explain how the original model arrives at estimates and how it extrapolates. If this takes too long and is too difficult, just make it top level and refer to the paper for details

Cookies: 

\begin{tabular}{p{0.4\textwidth}|p{0.4\textwidth}}
    %%%
    Technically necessary cookies & Technically non-necessary cookies \\
    \hline
    \hline
    \begin{itemize}
      \item I
      \end{itemize}
     &
      \begin{itemize}
      \item Sample Text
      \item Sample Text
    \end{itemize}\\
    %%%
    \begin{itemize}
      \item Sample Text
      \item Sample Text
      \end{itemize}
      &
      \begin{itemize}
      \item Sample Text
      \item Sample Text
    \end{itemize}
    %%%

    \end{tabular}