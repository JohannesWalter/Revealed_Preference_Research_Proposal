Subsection \ref{data coll} describes how the data is proposed to be collected and the kind of data that will be collected.
It will also contain a short overview of
browser cookies and their functionality. Section \ref{method} will present how
\textcite{goldin2020} identify their parameters of interest and the simplifications
that are possible when one has access to a richer dataset.

\subsection{Data} \label{data coll}
The process of data collection is inspired by \textcite{levy2020}. In this paper, the
author wrote a browser extension for Google's Chrome browser. Via a Facebook ad campaign,
8084 participants for a survey were attracted. At the end of the survey, they were offered a
small reward in exchange of installing the browser extension on their device \parencite[p. 9]{levy2020}.
2262 survey participants installed the extension, and 1835 kept the extension installed for at least two weeks.

This research proposal suggests to apply the same design. Participants can be attracted
via a Facebook or Google ad campaign. In a short survey, their age, gender and nationality can
be elicited. Afterwards, they are asked to install a small browser extension in exchange for
a small reward. The extension will use a short javascript code that writes a log file. This log file will keeps track whether a participant has 
visited a website that asks for cookie settings. It will also log whether
the default setting is to accept all cookies, or whether by default only the technically necessary cookies are selected.
Importantly, the extension needs not collect any information about the content of the visited websites. From a privacy point of view,
this design is less invasive than the one in \textcite{levy2020}. This observation motivates the believe that there is a sufficient amount of people who are willing
to participate in such a study.
%Chrome makes it easy to include extensions
% Talk about the money required

A hypertext transfer protocol (HTTP) cookie is a small piece of data that is sent from
a website and stored on a users device. For the purposes of this research proposal, cookies can be classified into two groups. First, there are
technically necessary cookies that ensure a website is working as intended. Such cookies could remember preceding user
interactions with the site, e.g. adding an item to a shopping cart, or logging in. Second, there are cookies that are not technically
necessary to ensure the website's functionality. A common example are for example third-party tracking cookies which record previously visited websites.
These non-essential third-party tracking cookies can generate long-term records of a user's browsing history and can therefore manifest a potential privacy concern.
Websites have an incentive to allow these cookies, as they receive third-party payments in return (common tracking cookies are e.g. Google AdSense or Facebook remarketing).  
The GDPR requires that all websites which target European Union member states ask their users for consent
before they are allowed to store non-essential cookies on a user's device. The host of a website still
has a choice on how to frame this question. Either non-essential cookies are unselected by default and if the
user clicks once on "accept" no tracking is possible. Or non-essential cookies are selected by default and if the user wants
to avoid them it is necessary to go into the website's privacy settings and unselect tracking cookies manually.
%order and at least one observation of each kind

\subsection{Method} \label{method}

This section will describe how \textcite{goldin2020} identify their model parameters. It will also contain a description
of how the parameter identification is simplified by the method proposed here.
To keep this section concise and within the scope of this proposal, only the main ideas are introduced here.
For a detailed discussion, see \textcite[p.2767]{goldin2020}.

In the first step, \textcite{goldin2020} focus on the consistent decision makers, i.e. the subgroup of individuals
whose decision does not depend on the frame. With a dataset as suggested above, it is trivial to find this group.
Once the consistent decision makers are found, using assumption A3), the consistency principle, the options they preferred can be found.

\textcite{goldin2020}'s method is also applicable to datasets that do not contain repeated observations for each individual.
It is worth noting, that \textcite{goldin2020} mention that there method can also improve datasets that do contain repeated observations.
In particular, this is the case when the \textit{order} in which the repeated observations are generated can itself be considered a framing. 
For instance, this could be an issue if researchers were to analyse survey data, where the questions were not sufficiently randomized. 
For the dataset that is described in section \ref{data coll} the order in which individuals see each question about cookies is, as described, assumed to follow a random process.

\textcite{goldin2020} continue by describing the conditions for which consistent decision makers preferences can be identified when each decision maker
is observed under a single frame. 

\begin{itemize}
    \item Proposition 1: $ \bar{Y}_C \equiv \bar{Y}(0)/(\bar{Y}(0) + 1 - \bar{Y}(1))$
    \item Proposition 1.1: Under assumptions A1 - A4, $ E[Y^*_i | C_i = 1] = \bar{Y}_C $.
    \item Proposition 1.2: Under assumptions A1 - A3, $ \bar{Y}_C \geq 1/2 \Rightarrow \bar{Y}_C \leq E[Y^*_i | C_i = 1] \leq 1$, and $\bar{Y}_C \leq 1/2 \Rightarrow 0 \leq E[Y^*_i| C_i = 1] \leq \bar{Y}_C$.
\end{itemize}

Proposition 1.1 follows from the insight that, given frame monotonicity, only consistent decision-makers choose against the frame \parencite[p. 2768]{goldin2020}.
Proposition 1.2 provides a partial identification result, as it defines upper and lower bounds for the fraction of frame consistent decision makers. The results under 
proposition 1.2 are robust to failures of frame monotonicity. In other words, it guarantees bounds in cases when there are decision makers
who decide against the frame, but are still in truth inconsistent. These decision makers are defined by \textcite[p. 2768]{goldin2020}
as frame defiers.

The logic of proposition 1 is illustrated in table \ref{table}, which is taken from \textcite[p. 2769]{goldin2020}. To understand this table, it is important to 
know the conditions that produced the data. Here it is assumed that the researchers observe answers under both types of framing. Yet each individual's answer is only
observed once under one type of framing.
\includegraphics{table.png} \label{table}
To the best of my understanding, the table contains a mistake. I believe that the description for the second row of the table should read
"Choose to enroll, opt-out regime, $ Y_i(1) = 1$" (instead in the version in the article, the description mistakenly contains a "not": "Choose not to enroll, opt-out regime, $ Y_i(1) = 1$")
Now, it becomes easy to see how the fraction of consistent decision makers can be found even if each decision makers is only observed under one frame.
In the example in table \ref{table}, 0 \% of observations fall into the top-right quadrant by assumption.
The top-left quadrant is the result of $1-\bar{Y}(1)$, where $\bar{Y}(1)$
is the fraction of decision makers who chose to enroll under the opt-out framing. This number is observed by the researchers.
The bottom left quadrant is the difference between the fraction of decision makers who enroll under opt-out framing and the fraction
of decision makers who enroll under opt-in framing. Both fractions are observed.
The bottom right quadrant is the fraction of decision makers who enroll under opt-in framing, which is again, observed.
Now the fraction of consistent decision makers who prefer option 1 can be calculated according to proposition 1. When assumption A4 is relaxed,
at least a lower bound can be calculated.
This way, despite the researcher not observing each decision maker under each framing, the preferences of the consistent decision makers can be recovered.

When a researcher has access to a dataset as proposed in this research proposal, the fraction of consistent decision makers preferring
option 1 can be calculated directly and could be compared to the result from \textcite{goldin2020}'s method (in table \ref{table} approximately 53\%). Thus, providing
a way to check the method for its validity in a certain setting.

In the next step, \textcite[p.2769]{goldin2020} describe several methods to carry the preferences of the consistent decision makers
over to the rest of the population. 

% 1. Identify the subgroup of consistent decision makers
% 2. Project from that subgroup onto the entire Population
%     2.1 Partial Identification
%     2.2 Selection on observables
%     2.3 Instruments

%     To illustrate
%     the notation with the privacy example from the introduction, let Yi indicate
%     whether i allows a company to use her data, Di 5 1 indicate the optout
%     regime, and Di 5 0 indicate the opt-in regime, so that Y ð1Þ 5 0:65
%     and Y ð0Þ 5 0:40.   

%A) Data
%Komplettes Deaktivieren der Cookies in Firefox oder Cookies deaktivieren für nutzungsbasierte Online-Werbung
%https://www.bild.de/wa/ll/bild-de/privater-modus-unangemeldet-54578900.bild.html

%By using our site, you acknowledge that you have read and understand our Cookie Policy, Privacy Policy, and our Terms of Service
%https://tex.stackexchange.com/questions/823/remove-ugly-borders-around-clickable-cross-references-and-hyperlinks

%I observe whether individual i accepts all cookies or only the technically necessary ones $(Y_{i})$, and whether the default is opt-in $ (D_{i} = 0) $ or opt-out $ (D_{i} = 1) $ at the date of visitng the website. 
%Through an additional survey, we also observe age, sex and race for each employee.

%B) Recovery of Consistent Preferences

%\begin{itemize}
%    \item Under assumptions A1 - A4, proposition 1 allows us to identify the preferences of the consistent visitors
%    \item A1) requires that the preferences over the cookie choice do not depend on opt-in or opt-out.
%    \item A2) Frame exogeneity
%\end{itemize}

%Under A)1 to A4), we identify the preferences of the consistent decision makers.
%What you get is the preference of the consistent decision-makers.

%C) Recovery of the Population Preferences

%Subsection 2) Identification and identification of popuilation preferences

%Explain how the original model arrives at estimates and how it extrapolates. If this takes too long and is too difficult, just make it top level and refer to the paper for details

%Cookies: 

%\begin{tabular}{p{0.4\textwidth}|p{0.4\textwidth}}
%    %%%
%    Technically necessary cookies & Technically non-necessary cookies \\
%    \hline
%    \hline
%    \begin{itemize}
%      \item I
%      \end{itemize}
%     &
%      \begin{itemize}
%      \item Sample Text
%      \item Sample Text
%    \end{itemize}\\
%    %%%
%    \begin{itemize}
%      \item Sample Text
%      \item Sample Text
%      \end{itemize}
%      &
%      \begin{itemize}
%      \item Sample Text
%      \item Sample Text
%    \end{itemize}
%    %%%

%   \end{tabular}