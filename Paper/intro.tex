

In stated preference studies people tend to express a high valuation for privacy, yet observed behavior is typically at odds with stated preferences.
This phenomenon has come to known in the literature as privacy paradox. Some have argued that this observation is 
merely illusory, since one can state a high valuation of privacy in general, but in a specific situation a cost-benefit analysis might 
lead people to discount privacy concerns \parencite[p. 2]{acquisti2015privacy}. This refutation falls short insofar, as it is a well documented result in behavioral economic research that people'
decision making capabilities are only in part rational. Especially in situations when people are uncertain about the consequences of their actions, when they are unsure about their preferences or when they 
are under time pressure, people often search for clues in their surrounding to provide guidance \parencite[p. 3]{acquisti2015privacy}. One possible source of 
orientation can stem from the specific way in which a decision is framed. Deriving revealed preferences in the presence of framing can be problematic, if people's preference 
is not consistent over the set of framings. \textcite{goldin2020} propose a new method to recover consistent population preferences even when framing effects are present. 

Their methodology rests on the insight that even though framing-consistent preferences cannot be observed on the \textit{individual level} in typical datasets, 
on the \textit{population level} the fraction of consistent decision-makers can be identified. From the subgroup of consistent decision-makers their approach
extrapolates to the entire population. Their method is described in detail in section \ref{model}.
% Describe in more detial ithe method and what data is neeeded and that the problem is

This research draft proposes two innovations. First, it is proposed to collect an online dataset recording people's decisions about browser cookie settings via a type of data collection
akin to the one in \textcite{levy2020}. \textcite{goldin2020}'s method can subsequently be applied to the analysis of this dataset. The second innovation is methodological in nature.
The dataset collected online will contain repeated observations of the same individual both under an ad-tracking cookie opt-in framing and an ad-tracking cookie opt-out framing. 
This additional information can be used to overcomes the central limitation in most datasets. This means that the necessity for \textcite{goldin2020}'s model vanishes. 
Instead, the subset of framing-consistent people can be observed on the individual level. Once these individuals are identified, one can again extrapolate to the entire population, as
is done in \textcite{goldin2020}'s method. Finally, the results from the first approach and the second approach can be compared, yielding an empirical test of 
\textcite{goldin2020}'s method.

To understand the 
data
GDPR introduced mandatory choice

What would ideal results of this research look like? If the proposed research succeeds it find whether \textcite{goldin2020}'s new method generates estimates that are 
qualitatively similar and quantitatively close to estimates based on repeated observations of the same individual.
By doing so, it will produce estimates of people's privacy preferences which are robust with respect to framing. Here two scenarios are conceivable: Either it will
be found that   From a policy point of view, the latter result could possibly be interpreted to imply that the rules governing cookie default settings would have to be
further tightened, such that websites default setting is to only use technically necessary cookies. Ad-tracking cookies would then mandatorily be opt-in.



Policy implications

This research lies in the intersection of three strands of literature, each of which it will attempt to contribute to. First, it will contribute to the 
literature on measuring preferences for and attitudes towards privacy. Both in the social and behavioral sciences There is a large body of literature that examines the 
nature of privacy preferences. This research will contribute by uncovering preferences for online privacy settings in particular. Second, by carrying \textcite{goldin2020}'s
model into a new setting and confronting it with an empirical test, this research will contribute to the literature on revealed preferences under framing.
And finally, there is a growing literature on the effects of the GDPR. The change in laws regarding cookie settings is what enables this research in the first place.
The results of this study will have practical implications for the question whether even stricter privacy regulations seem justified.




mention the types of framing effects and that the relevant one here is a default type
mention that their model does it with a 401k plan that doesnt allow for repeated observation. Our setting and method allows for repeated observation though




This causeses a problem for revelaeded pref analy if deciciosn depend on framing  paper by dings  in context of digital bums
what is their overall idea:
How Im gonna use it 
data I need and how to get 
contribution to the literature


methodoloogical innovation with my idea




Idea: 
\begin{itemize}
    \item Setting: GDPR introduced choice on cookies. Cookies choice depends on framing. This new model in the JPE can factor in framing. We can then derive people's revealed preference for privacy
    \item Data: Collect browsing behavior from random sample. Requirement: Browser add-on. I can ref
    \item Create appendix chapter with a table detailing the different possible browser cookie settings and images of a couple of cookie setting screenshots.
\end{itemize}


Literatures touched: revealed preferences under framing, privacy evaluation, effects of GDPR.
Previous literature suggests that there must be framing effects -> Madrian and Shea
Lierature also belongs to how does the GDPR affect people and firms?
Policy Implicactions: GDPR wants to privacy protect citizens. If design fails, implication is 
that all sites should default to only techn necessary
What would ideal results look like?

\cite{acquisti2015privacy}  mention online example even though they go through with the 401k plan example