Idea: 
\begin{itemize}
    \item Topic: Revealed preferences and privacy
    \item Advantage of this approach: connects RP and my field of Digital Econ
    \item Setting: GDPR introduced choice on cookies. Cookies choice depends on framing. This new model in the JPE can factor in framing. We can then derive people's revealed preference for privacy
    \item Data: Collect browsing behavoir from random sample. Requirement: Browser add-on. I can refer to the Paper from the conference.
    \item Model: From the JPE Paper, that's gonna be nice
    \item Intro: Science article, put it in general  context of revealed preference research on privacy. If it doesn't exist yet, great (but unlikely). If it does already exist, I'd wager nobody so far has considered the effects of framing with this new model.
    \item Literature: Does this work? \parencite{acquisti2015privacy} And this? \parencite{levy2020}
\end{itemize}


Literatures touched: revealed preferences under framing, privacy evaluation, effects of GDPR.
Previous literature suggests that there must be framing effects -> Madrian and Shea
Lierature also belongs to how does the GDPR affect people and firms?
Policy Implicactions: GDPR wants to privacy protect citizens. If design fails, implication is 
that all sites should default to only techn necessary
What would ideal results look like?

\cite{acquisti2015privacy}  mention online example even though they go through with the 401k plan example