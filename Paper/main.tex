\documentclass[12pt,a4paper]{article}
\usepackage[utf8]{inputenc}
\usepackage[english]{babel}
\usepackage{url}
\usepackage{hyperref}
\usepackage{amsmath}
\usepackage{amsfonts}
\usepackage{amssymb}
\usepackage{graphicx}
\usepackage[left=2cm,right=2cm,top=2cm,bottom=2cm]{geometry}
\usepackage[style=authoryear,sorting=nty]{biblatex}
\addbibresource{literatur.bib}
\author{Johannes Walter}
\newcommand{\indep}{\perp \!\!\! \perp}

\usepackage{xcolor}
\hypersetup{
    colorlinks,
    linkcolor={red!50!black},
    citecolor={blue!50!black},
    urlcolor={blue!80!black}
}

\begin{document}

\begin{titlepage}
    \centering
    \vspace{5cm}
    {\scshape Centre for European Economics Research \par}
    \vspace{0.5cm}
    {\scshape Research Proposal Summer School Revealed Preferences \par}
    \vspace{1.5cm}
    {\scshape\LARGE Revealed Preferences under Framing: Users valuation of privacy \par}
    \vspace{2cm}
    {\itshape Johannes Walter \par}
    \vspace{1cm}
    \begin{abstract}
    Diese Dokumentation enth"alt eine sortierte Liste der wichtigsten
    \LaTeX--Befehle. Die einzelnen Listeneintr"age sind untereinander
    durch viele Querverweise verkettet, die ein Auffinden inhaltlich
    zusammengeh"origer Informationen erheblich erleichtern.
    \end{abstract}
    \vfill
    Summer School Revealed Preferences \par 
    Nicolai Kuminoff 
    \vfill

    % Bottom of page
    {\large \today\par}
\end{titlepage}


\section{Introduction}
\label{introduction}


In stated preference studies people tend to express a high valuation for privacy, yet observed behavior is typically at odds with these stated preferences.
This phenomenon has come to be known in the literature as privacy paradox \parencite{gerber2018}. Some have argued that this observation is 
merely illusory since an individual can state a high valuation of privacy in general, but in a specific situation, a cost-benefit analysis might 
lead people to discount privacy concerns \parencite[p. 2]{acquisti2015privacy}. This refutation falls short insofar, as it is a well documented result in behavioral
economic research that people's
decision making capabilities are only in part rational. Especially in situations when people are uncertain about the consequences of their actions, when they are unsure about their preferences, or when they 
are under time pressure, people often search for clues in their surrounding to provide guidance \parencite[p. 3]{acquisti2015privacy}. One possible source of 
orientation can stem from the specific way in which a decision is framed. Framing effects, in turn, introduce a difficulty when researchers are interested in learning about people's true preferences.
Deriving revealed preferences in the presence of framing can be problematic if people's preference 
is not consistent over the set of framings. \textcite{goldin2020} propose a new method to identify the fraction of consistent decision makers
and to extrapolate from them to the entire population. A decision maker is consistent when her decision is not influenced by the frame. Preferences can then then be recovered even when framing effects are present, as if the whole population was comprised of framing-consistent decision makers.

Their methodology rests on the insight that even though framing-consistent preferences cannot be observed on the \textit{individual level} in typical datasets (i.e in datasets where researchers
observe a respondent's answer generated under only one framing), 
on the \textit{population level}, the fraction of consistent decision-makers can be identified if researchers see answers from all framings. From the subgroup of consistent decision makers their approach
extrapolates to the entire population. Their method is described in detail in section \ref{model}.
% Describe in more detial ithe method and what data is neeeded and that the problem is
\\
\\
This research proposal suggests to build on the results obtained by \textcite{goldin2020}. It suggests to collect a dataset that can be used to test their method. At the
same time, the results from the analysis of this dataset will produce framing-consistent estimates of users' privacy setting preferences. The test of \textcite{goldin2020}'s method
will constitute a methodological contribution to the literature on revealed preferences under framing. %literature on preferences is mentioned in two consecutive statements, is that okay?
The preference estimates will contribute to the literature on
privacy preferences and will have implications for policy makers concerned with the question of security of personal data on the internet.
\\
\\
In particular, it is proposed to collect a dataset online recording people's decisions about browser cookie settings via a type of data collection
akin to the one in \textcite{levy2020}. \textcite{levy2020} wrote a browser add-on for Google's Chrome browser, which tracked in an experiment
the participants' visited websites for two weeks. It is proposed to write a similar browser add-on to track people's decision about browser cookie settings.
According to the General Data Protection Regulation (GDPR) users who visit a website from within the European Union or the European Economic Area must be asked for their
stated consent on storing browser cookies \parencite{GDPR}. For the intents of this research proposal, one can divide browser cookies into two groups.
First, there are essential cookies that are necessary to guarantee the website's functionality. Second, there are third-party ad-tracking cookies. 
A website's host still has a monetary incentive to nudge users to allow for the ad-tracking cookies. This can be done by choosing the default cookie settings.

The dataset collected online will contain repeated observations of the same individual both under an ad-tracking cookie opt-in framing and an ad-tracking cookie opt-out framing.
This additional information can be used to overcome a central limitation in many datasets (where individuals are observed only under one frame).
The subset of framing-consistent people can therefore be observed on the individual level. Once these individuals are identified, one can again extrapolate to the entire population, as
is done in \textcite{goldin2020}'s method. Finally, the results from the first approach and the second approach can be compared, yielding an empirical test of 
\textcite{goldin2020}'s method.

What would ideal results of this research look like? If the proposed research succeeds, it finds whether \textcite{goldin2020}'s new method generates estimates that are 
qualitatively similar and quantitatively close to estimates based on repeated observations of the same individual.
By doing so, it will produce estimates of people's privacy preferences which are robust with respect to framing. Here two scenarios are conceivable: It will
be found that a large proportion or even a majority of users do have a preference for heightened privacy, i.e. a preference against non essential ad-tracking cookies.
Alternatively, it could be found that users do have a preference for ad-tracking cookies, e.g. because allowing for this type of cookie increases the relevance of online advertisement.
From a policy point of view, the former result could possibly be interpreted to imply that the rules governing cookie default settings would have to be
further tightened, such that websites default setting is to only use technically necessary cookies by default. Ad-tracking cookies would then mandatorily be opt-in. Either way,
the analysis will allow to examine if there are interesting sources of heterogeneity in the data, potentially revealing previously unknown relationships between
users and their privacy attitudes.
\\
\\
The remainder of this research proposal is structured as follows. Section \ref{basic notation} will introduce the basic idea and notation of \textcite{goldin2020}'s method. Section
\ref{ass} will introduce the assumptions that form the basis for their identification strategy. Section \ref{data coll} will describe how and what kind of data is proposed to be collected.
Section \ref{method} will describe the identification strategy of \textcite{goldin2020}'s method. Section \ref{sum} concludes.







% mention that their model does it with a 401k plan that doesnt allow for repeated observation. Our setting and method allows for repeated observation though




% This causeses a problem for revelaeded pref analy if deciciosn depend on framing  paper by dings  in context of digital bums
% what is their overall idea:
% How Im gonna use it 
% data I need and how to get 
% contribution to the literature


% methodoloogical innovation with my idea




% Idea: 
% \begin{itemize}
%     \item Setting: GDPR introduced choice on cookies. Cookies choice depends on framing. This new model in the JPE can factor in framing. We can then derive people's revealed preference for privacy
%     \item Data: Collect browsing behavior from random sample. Requirement: Browser add-on. I can ref
%     \item Create appendix chapter with a table detailing the different possible browser cookie settings and images of a couple of cookie setting screenshots.
% \end{itemize}


% Literatures touched: revealed preferences under framing, privacy evaluation, effects of GDPR.
% Previous literature suggests that there must be framing effects -> Madrian and Shea
% Lierature also belongs to how does the GDPR affect people and firms?
% Policy Implicactions: GDPR wants to privacy protect citizens. If design fails, implication is 
% that all sites should default to only techn necessary
% What would ideal results look like?

% \cite{acquisti2015privacy}  mention online example even though they go through with the 401k plan example

\section{Economic Model}
\label{model}
The description of this model follows chapter one of \textcite{acquisti2015privacy}. A fully detailed description 
of the model can be found there and is beyond the scope of this research proposal. Here, only the main concepts 
necessary for understanding this proposal are presented.

A decision maker i chooses from a binary decision set $ \textbf{S} = \lbrace 0,1 \rbrace  $ und two possible frames $ D_{i} \in \lbrace 0,1 \rbrace $. 
In the context of this research proposal, the decision coded with a $0$ 
could refer to an individual's decision to accept only technically necessary cookies and 
the decision coded with a $ 1 $ could refer to an individual's decision 
to allow for non technically necessary cookies. Frame $ 0 $ could be the situation where the default setting is such that 
technically necessary cookies are the only pre-chosen ones and the user would have to actively engage in clicking on all non-technically necessary 
cookies she wishes to allow. Frame $ 1 $ could then refer to the situation 
where in the default setting both types of cookies are pre-chosen and the user accepts both types with one click. We can adapt a notation that is akin to what many researchers like to use in a potential outcome setting. $ Y_{i}(0) $ and $ Y_{i}(0) $ denotes then the decision individual i makes under frame $ D_{i} = 0 $ and $ D_{i} = 1 $, respectively.
Decision makers are assumed to have strict ordinal preferences over the set of available options. $ Y^{*}_{i} \in \lbrace 0,1 \rbrace $ denotes the most preferred option.
Each decision maker is characterized by a vector of random variables 
$ (Y_i(0), Y_i(1), D_i, Y^*_i) $, which are drawn from an underlying population distribution.
For each $ i $, the researcher observes the pair $ (Y_i, D_i) $, where 
$ Y_i = Y_i(0)D_i + Y_i(1)(1-D_i)$. \textcite{goldin2020} assume for their model 
that the researcher does not observe $ Y^*_i $ and only observes one of $ Y_i(0)$ and $ Y_i(1)$, depending on the frame $D_i$.

The data that is suggested to be collected in this proposal deviates
in a crucial way from this assumption. In contrast to the datasets \textcite{goldin2020}
have in mind, the dataset in this proposal will contain \textit{repeated} observations of the same individual. 
The number of observations for each $ i $ can be denoted with a subscript 
$ k \in \lbrace 1,\dots, K \rbrace $ such that $ Y_{i, k} $ is observed. 
Importantly, here it is assumed that $ K $ is sufficiently large to allow for at 
least one observation of each framing $ D_{i} \in \lbrace 0,1 \rbrace $.

\textcite{goldin2020} continue their description of the model as follows: Each
decision maker can choose either consistently, i.e. the same choice under each frame,
or choose in a way that is responsive to the frame. Consistency is denoted by $ C_i = 1 iff Y_i(0) = Y_i(1) else C_i = 0 $.
Again, they assume that each $ i $ is observed only under one frame, such that $C_i$ is not observed.
In the context of this proposal, as described above, it is assumed that each $ i $ is observed under both frames, such that
$ C_i $ is identified. This difference is crucial for one of the main contributions of this proposal.
The additional information in this dataset allows to calculate two sets of results. For the first one, the additional information in this dataset is not used.
The analysis will proceed under the assumption of \textcite{goldin2020}. The second set of results will be obtained using the full amount of information
in the dataset. As such, the "ground-truth" that remains hidden under \textcite{goldin2020}'s assumptions, is observed. In other words, with the dataset proposed 
here, it will be able to identify all individuals who make consistent choices, independent of the frame. 
Having these two sets of results, one can compare the estimates from both to arrive at an assessment of the quality of \textcite{goldin2020}'s model accuracy.

Four +1 assumptions real quick


\begin{itemize}
    \item Assumption A1 (Frame separability) For all $ i $, $ Y^*_{i} $ does not depend on $ D $.
\end{itemize}



\section{Data and Methods}
\label{data and methods}
Subsection 1) Data collection
describe how he did it and what kind of data you need to collect (i.e. recognize when asked for cookies and what the default cookie setting is)

Also describe a little bit how cookies actually work

What kind of data is required?




\begin{itemize}
    \item Data needs to have an opt-in and an opt-out possible framing. This should be given in the cookie context: 
    \begin{itemize}
        \item What does the GDPR say exactly?
        \item Either it says all cookies always must be disabled by default \dots 
        \item Or it varies only by how the website presents the choice. What are the options? 
    \end{itemize}
    \item I need to observe whether individual i chooses only the technically necessary cookies or all the cookies
    \item I also need to observe how the choice is presented: 
\end{itemize}

A) Data
Komplettes Deaktivieren der Cookies in Firefox oder Cookies deaktivieren für nutzungsbasierte Online-Werbung
https://www.bild.de/wa/ll/bild-de/privater-modus-unangemeldet-54578900.bild.html

By using our site, you acknowledge that you have read and understand our Cookie Policy, Privacy Policy, and our Terms of Service
https://tex.stackexchange.com/questions/823/remove-ugly-borders-around-clickable-cross-references-and-hyperlinks

I observe whether individual i accepts all cookies or only the technically necessary ones $(Y_{i})$, and whether the default is opt-in $ (D_{i} = 0) $ or opt-out $ (D_{i} = 1) $ at the date of visitng the website. 
Through an additional survey, we also observe age, sex and race for each employee.

B) Recovery of Consistent Preferences

\begin{itemize}
    \item Under assumptions A1 - A4, proposition 1 allows us to identify the preferences of the consistent visitors
    \item A1) requires that the preferences over the cookie choice do not depend on opt-in or opt-out.
    \item A2) Frame exogeneity
\end{itemize}

Under A)1 to A4), we identify the preferences of the consistent decision makers.
What you get is the preference of the consistent decision-makers.

C) Recovery of the Population Preferences

Subsection 2) Identification and identification of popuilation preferences

Explain how the original model arrives at estimates and how it extrapolates. If this takes too long and is too difficult, just make it top level and refer to the paper for details

Cookies: 

\begin{tabular}{p{0.4\textwidth}|p{0.4\textwidth}}
    %%%
    Technically necessary cookies & Technically non-necessary cookies \\
    \hline
    \hline
    \begin{itemize}
      \item I
      \end{itemize}
     &
      \begin{itemize}
      \item Sample Text
      \item Sample Text
    \end{itemize}\\
    %%%
    \begin{itemize}
      \item Sample Text
      \item Sample Text
      \end{itemize}
      &
      \begin{itemize}
      \item Sample Text
      \item Sample Text
    \end{itemize}
    %%%

    \end{tabular}

\printbibliography


\end{document}