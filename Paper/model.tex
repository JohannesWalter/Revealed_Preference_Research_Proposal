The description of this model follows chapter one of \textcite{acquisti2015privacy}. A fully detailed description 
of the model can be found there and is beyond the scope of this research proposal. Here, only the main concepts 
necessary for understanding this proposal are presented.

A decision maker i chooses from a binary decision set $ \textbf{S} = \lbrace 0,1 \rbrace  $ und two possible frames $ D_{i} \in \lbrace 0,1 \rbrace $. 
In the context of this research proposal, the decision coded with a $0$ 
could refer to an individual's decision to accept only technically necessary cookies and 
the decision coded with a $ 1 $ could refer to an individual's decision 
to allow for non technically necessary cookies. Frame $ 0 $ could be the situation where the default setting is such that 
technically necessary cookies are the only pre-chosen ones and the user would have to actively engage in clicking on all non-technically necessary 
cookies she wishes to allow. Frame $ 1 $ could then refer to the situation 
where in the default setting both types of cookies are pre-chosen and the user accepts both types with one click. We can adapt a notation that is akin to what many researchers like to use in a potential outcome setting. $ Y_{i}(0) $ and $ Y_{i}(0) $ denotes then the decision individual i makes under frame $ D_{i} = 0 $ and $ D_{i} = 1 $, respectively.
Decision makers are assumed to have strict ordinal preferences over the set of available options. $ Y^{*}_{i} \in \lbrace 0,1 \rbrace $ denotes the most preferred option.
Each decision maker is characterized by a vector of random variables 
$ (Y_i(0), Y_i(1), D_i, Y^*_i) $, which are drawn from an underlying population distribution.
For each $ i $, the researcher observes the pair $ (Y_i, D_i) $, where 
$ Y_i = Y_i(0)D_i + Y_i(1)(1-D_i)$. \textcite{goldin2020} assume for their model 
that the researcher does not observe $ Y^*_i $ and only observes one of $ Y_i(0)$ and $ Y_i(1)$, depending on the frame $D_i$.

The data that is suggested to be collected in this proposal deviates
in a crucial way from this assumption. In contrast to the datasets \textcite{goldin2020}
have in mind, the dataset in this proposal will contain \textit{repeated} observations of the same individual. 
The number of observations for each $ i $ can be denoted with a subscript 
$ k \in \lbrace 1,\dots, K \rbrace $ such that $ Y_{i, k} $ is observed. 
Importantly, here it is assumed that $ K $ is sufficiently large to allow for at 
least one observation of each framing $ D_{i} \in \lbrace 0,1 \rbrace $.

\textcite{goldin2020} continue their description of the model as follows: Each
decision maker can choose either consistently, i.e. the same choice under each frame,
or choose in a way that is responsive to the frame. Consistency is denoted by $ C_i = 1 iff Y_i(0) = Y_i(1) else C_i = 0 $.
Again, they assume that each $ i $ is observed only under one frame, such that $C_i$ is not observed.
In the context of this proposal, as described above, it is assumed that each $ i $ is observed under both frames, such that
$ C_i $ is identified. This difference is crucial for one of the main contributions of this proposal.
The additional information in this dataset allows to calculate two sets of results. For the first one, the additional information in this dataset is not used.
The analysis will proceed under the assumption of \textcite{goldin2020}. The second set of results will be obtained using the full amount of information
in the dataset. As such, the "ground-truth" that remains hidden under \textcite{goldin2020}'s assumptions, is observed. In other words, with the dataset proposed 
here, it will be able to identify all individuals who make consistent choices, independent of the frame. 
Having these two sets of results, one can compare the estimates from both to arrive at an assessment of the quality of \textcite{goldin2020}'s model accuracy.

Four +1 assumptions real quick


\begin{itemize}
    \item Assumption A1 (Frame separability) For all $ i $, $ Y^*_{i} $ does not depend on $ D $.
\end{itemize}

